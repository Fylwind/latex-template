\documentclass[fleqn, 12pt]{article}
\usepackage{amsmath}                    % AMS math utilities
%\usepackage{amssymb}                    % AMS symbols
\usepackage{booktabs}                   % Tables
%\usepackage{empheq}                     % Boxed equations
\usepackage{fancybox}
\usepackage{fancyhdr}                   % Headers & footers
\usepackage{isomath}                    % ISO math formatting
\usepackage{mdframed}                   % Framed boxes
\usepackage{microtype}                  %
\usepackage{parskip}                    % No paragraph indent
\usepackage{siunitx}                    % Support for units
\usepackage{subcaption}                 % For subfigures
\usepackage{subdepth}                   % Make subscripts appear at same place
\usepackage{tikz}                       % For drawing stuff

% Font settings
\ifcase 1

  % Note: font is somewhat buggy.
  \usepackage{newpxtext}
  \usepackage[slantedGreek, vvarbb]{newpxmath}
  \renewcommand{\dagger}{\text{\dag}}

\or

  \usepackage[noDcommand, nott, slantedGreeks]{kpfonts}
  \renewcommand*\ttdefault{txtt}
  \usepackage[T1]{fontenc}

\fi

% Bold symbols package (note: may overload the number of math alphabets)
% (!) Must load *after* the fonts!
%\newcommand{\hmmax}{0}
%\newcommand{\bmmax}{2}
\usepackage{bm}

% Set height of header
\setlength{\headheight}{15pt}

\usepackage[includeheadfoot]{geometry}
\pagestyle{fancy}
\usetikzlibrary{calc, shapes}

% Allow large matrices
\setcounter{MaxMatrixCols}{50}

% Fix the spacing bug with delimiters
\let\originalleft\left
\let\originalright\right
\renewcommand{\left}{\mathopen{}\mathclose\bgroup\originalleft}
\renewcommand{\right}{\aftergroup\egroup\originalright}

% Numbering
\numberwithin{equation}{section}
\renewcommand{\theequation}{\textsf{\footnotesize
    \ensuremath{\diamond}\arabic{equation}}}
\definecolor{answercolor}{rgb}{0, .2, .7}
\newenvironment{answer}{\begingroup\color{answercolor}}{\endgroup}

% Special boxed equation
\definecolor{hicolor}{rgb}{1, .9, .6}
\newcommand{\himath}[1]{\colorbox{hicolor}{\ensuremath{\displaystyle
      #1}}}
\definecolor{hicolordark}{rgb}{.8, .3, 0}
\newcommand{\hitext}[1]{\textcolor{hicolordark}{#1}}
\definecolor{resultboxcolor}{rgb}{.8, 1, .7}
\newcommand{\resultmath}[1]{\colorbox{resultboxcolor}{\ensuremath{\displaystyle
      #1}}}
\definecolor{resultcolordark}{rgb}{.4, .8, 0}
\newcommand{\resulttext}[1]{\textcolor{resulcolordark}{#1}}
\newmdenv[
    backgroundcolor=resultboxcolor,
    innertopmargin=15,
    linewidth=0
]{result}
% \newcommand{\finalbox}[1]{
%   \colorbox{resultboxcolor}{\hspace{1em}#1\hspace{1em}}}
% \newenvironment{finalalign*}{\begingroup\empheq[box=\finalbox
%   ]{align*}}{\endempheq\endgroup}
% \newenvironment{finalalign}{\begingroup\empheq[box=\finalbox
%   ]{align}}{\endempheq\endgroup}

% Mathematical constants (typeset in upright)
\newcommand{\E}{\mathrm e}
\newcommand{\I}{\mathrm i}
\newcommand{\PI}{\piup}

% Alter the formatting for vectors
\newcommand{\vc}[1]{\boldsymbol{#1}}%\vectorsym{#1}}

% Bra-ket notation
\newcommand{\bra}[1]{\left\langle#1\right\rvert}
\newcommand{\ket}[1]{\left\lvert#1\right\rangle}
\newcommand{\braket}[2]{\left\langle#1\middle|#2\right\rangle}
\newcommand{\brakket}[3]{\left\langle#1\middle|#2\middle|#3\right\rangle}

% Average operator (using angle brackets)
\providecommand{\avg}[1]{\left\langle#1\right\rangle}
\providecommand{\abs}[1]{\left\lvert#1\right\rvert}

\providecommand{\bigO}[1]{\mathcal{O}\left(#1\right)}

\newcommand{\normalorder}[1]{\left\rmoustache#1\right\lmoustache}

% The \lcontr and \rcontr commands are used to display Wick contractions (the
% left and right parts, respectively).  The syntax is given by:
%
%     \lcontr{INDEX}{OPERATOR}
%     \rcontr[GAP]{INDEX}{OPERATOR}
%
% where:
%
% * INDEX is a unique number used to both identify and specify the height of
%   the contraction.  The number only needs to be unique within the same
%   product.  For best display, use natural numbers starting from one.
%
% * OPERATOR is the operator, e.g. \hat{a}.
%
% * GAP is the small gap between the operator and the end of the line.
%   Default value is 0.03.
%
% Note that this will require two TeX passes to be displayed properly.
%
\newcommand{\lcontr}[2]{
  \tikz[baseline = (contraction-#1-left.base), remember picture]
  \node[inner sep = 0] (contraction-#1-left) {\ensuremath{#2}};
}
\newcommand{\rcontr}[3][.03]{
  \tikz[baseline = (contraction-#2-right.base), remember picture]
  \node[inner sep = 0] (contraction-#2-right) {\ensuremath{#3}};
  \pgfmathsetmacro{\contractionheight}{#2 * .1}
  \pgfmathparse{#2 > 0 ? "north" : "south"}
  \let\contractiondirection\pgfmathresult
  \pgfmathparse{#2 > 0 ? #1 : -#1 - .1}
  \let\contractionoffset\pgfmathresult
  \begin{tikzpicture}[overlay, remember picture, thick]
    \draw    ($ (contraction-#2-left.\contractiondirection)
                + (0, \contractionoffset) $)
          -- ++($ (0, \contractionheight) $)
          -| ($ (contraction-#2-right.\contractiondirection)
                + (0, \contractionoffset) $);
  \end{tikzpicture}
}

% The macros \create and \destroy represent creation and annihilation
% operators.  Each has an optional argument used to specify the subscript
% (i.e. quantum numbers).
%
% The \l* and \r* creation and annihilation operators are used for displaying
% the left and right parts of Wick contractions, respectively.  It requires a
% mandatory numeric argument used to indentify the contraction.  See docs for
% \lcontr and \rcontr for details.
\newcommand{\create}[1][]{\hat{a}^\dagger_{#1}}
\newcommand{\destroy}[1][]{\hat{a}^{}_{#1}}
\newcommand{\lcreate}[2][]{\lcontr{#2}{\hat{a}}^\dagger_{#1}}
\newcommand{\ldestro}[2][]{\lcontr{#2}{\hat{a}}^{}_{#1}}
\newcommand{\ldestroy}[2][]{\lcontr{#2}{\hat{a}}^{}_{#1}}
\newcommand{\rcreate}[2][]{\rcontr{#2}{\hat{a}}^\dagger_{#1}}
\newcommand{\rdestroy}[2][]{\rcontr{#2}{\hat{a}}^{}_{#1}}

\newcommand{\Kroneckerdelta}{\deltaup}
\newcommand{\LeviCivita}{\epsilonup}

%% import Data.Char
%% greek = ["alpha", "beta", "gamma", "delta", "epsilon",
%%          "zeta", "eta", "theta", "iota", "kappa", "lambda",
%%          "mu", "nu", "xi", "omicron", "pi", "rho", "sigma",
%%          "tau", "upsilon", "phi", "chi", "psi", "omega"]
%% main = let f x = "\\providecommand{\\" ++ x ++ "up}{\\up" ++ x ++ "}\n" ++
%%                  "\\providecommand{\\up" ++ x ++ "}{\\" ++ x ++ "}\n" in
%%        putStr $ concat [f x ++ f (toUpper (head x) : tail x) | x <- greek]
\providecommand{\alphaup}{\upalpha}
\providecommand{\upalpha}{\alpha}
\providecommand{\Alphaup}{\upAlpha}
\providecommand{\upAlpha}{\Alpha}
\providecommand{\betaup}{\upbeta}
\providecommand{\upbeta}{\beta}
\providecommand{\Betaup}{\upBeta}
\providecommand{\upBeta}{\Beta}
\providecommand{\gammaup}{\upgamma}
\providecommand{\upgamma}{\gamma}
\providecommand{\Gammaup}{\upGamma}
\providecommand{\upGamma}{\Gamma}
\providecommand{\deltaup}{\updelta}
\providecommand{\updelta}{\delta}
\providecommand{\Deltaup}{\upDelta}
\providecommand{\upDelta}{\Delta}
\providecommand{\epsilonup}{\upepsilon}
\providecommand{\upepsilon}{\epsilon}
\providecommand{\Epsilonup}{\upEpsilon}
\providecommand{\upEpsilon}{\Epsilon}
\providecommand{\zetaup}{\upzeta}
\providecommand{\upzeta}{\zeta}
\providecommand{\Zetaup}{\upZeta}
\providecommand{\upZeta}{\Zeta}
\providecommand{\etaup}{\upeta}
\providecommand{\upeta}{\eta}
\providecommand{\Etaup}{\upEta}
\providecommand{\upEta}{\Eta}
\providecommand{\thetaup}{\uptheta}
\providecommand{\uptheta}{\theta}
\providecommand{\Thetaup}{\upTheta}
\providecommand{\upTheta}{\Theta}
\providecommand{\iotaup}{\upiota}
\providecommand{\upiota}{\iota}
\providecommand{\Iotaup}{\upIota}
\providecommand{\upIota}{\Iota}
\providecommand{\kappaup}{\upkappa}
\providecommand{\upkappa}{\kappa}
\providecommand{\Kappaup}{\upKappa}
\providecommand{\upKappa}{\Kappa}
\providecommand{\lambdaup}{\uplambda}
\providecommand{\uplambda}{\lambda}
\providecommand{\Lambdaup}{\upLambda}
\providecommand{\upLambda}{\Lambda}
\providecommand{\muup}{\upmu}
\providecommand{\upmu}{\mu}
\providecommand{\Muup}{\upMu}
\providecommand{\upMu}{\Mu}
\providecommand{\nuup}{\upnu}
\providecommand{\upnu}{\nu}
\providecommand{\Nuup}{\upNu}
\providecommand{\upNu}{\Nu}
\providecommand{\xiup}{\upxi}
\providecommand{\upxi}{\xi}
\providecommand{\Xiup}{\upXi}
\providecommand{\upXi}{\Xi}
\providecommand{\omicronup}{\upomicron}
\providecommand{\upomicron}{\omicron}
\providecommand{\Omicronup}{\upOmicron}
\providecommand{\upOmicron}{\Omicron}
\providecommand{\piup}{\uppi}
\providecommand{\uppi}{\pi}
\providecommand{\Piup}{\upPi}
\providecommand{\upPi}{\Pi}
\providecommand{\rhoup}{\uprho}
\providecommand{\uprho}{\rho}
\providecommand{\Rhoup}{\upRho}
\providecommand{\upRho}{\Rho}
\providecommand{\sigmaup}{\upsigma}
\providecommand{\upsigma}{\sigma}
\providecommand{\Sigmaup}{\upSigma}
\providecommand{\upSigma}{\Sigma}
\providecommand{\tauup}{\uptau}
\providecommand{\uptau}{\tau}
\providecommand{\Tauup}{\upTau}
\providecommand{\upTau}{\Tau}
\providecommand{\upsilonup}{\upupsilon}
\providecommand{\upupsilon}{\upsilon}
\providecommand{\Upsilonup}{\upUpsilon}
\providecommand{\upUpsilon}{\Upsilon}
\providecommand{\phiup}{\upphi}
\providecommand{\upphi}{\phi}
\providecommand{\Phiup}{\upPhi}
\providecommand{\upPhi}{\Phi}
\providecommand{\chiup}{\upchi}
\providecommand{\upchi}{\chi}
\providecommand{\Chiup}{\upChi}
\providecommand{\upChi}{\Chi}
\providecommand{\psiup}{\uppsi}
\providecommand{\uppsi}{\psi}
\providecommand{\Psiup}{\upPsi}
\providecommand{\upPsi}{\Psi}
\providecommand{\omegaup}{\upomega}
\providecommand{\upomega}{\omega}
\providecommand{\Omegaup}{\upOmega}
\providecommand{\upOmega}{\Omega}

% Misc. commands
\def\reals{\mathbb{R}}
\def\integers{\mathbb{Z}}
\def\up{\uparrow}
\def\down{\downarrow}
\newcommand{\param}[1]{{\scriptstyle #1 =}}

% Differential operators
\def\D{\operatorname{d \!}}
\def\deriv{\operatorname{D \!}}

% Real/imaginary part operators: \Re & \Im.  The original symbols (\Re & \Im)
% are renamed to \Resymb and \Imsymb respectively.
\let\Resymb\Re
\let\Imsymb\Im
\def\Re{\operatorname{Re}}
\def\Im{\operatorname{Im}}
\def\tr{\operatorname{tr}}

% Misc. operators
\DeclareMathOperator\sgn{sgn}
\DeclareMathOperator\trace{tr}

% Spacing used for aligning the start of an equation, e.g.
%       2 (a + b)
%     = 2 a + 2 b
% Here, insert \eqbegin at the start of the equation.
\def\eqbegin{\mathrel{\phantom{=}}}

% Spacing used for continuing a line-break in an equation, e.g.
%     = 2 (a + b)
%       - 3 (x + y)
% Here, insert \eqcont right before the minus sign.
\def\eqcont{\eqbegin{}}

% Contracted index
\definecolor{ctrcolor}{rgb}{.4, .4, .4}
\newcommand{\ctr}[1]{\textcolor{ctrcolor}{#1}}
